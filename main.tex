\documentclass{article}
\usepackage[utf8]{inputenc}
\usepackage{hyperref}


\title{Requirements Engineering - Portfolio}
\author{Marko Milardić}
\date{September 2021}

\begin{document}

\maketitle
\section{Questions}

\begin{enumerate}
\item List the different definitions you found, the ones that were part of the course material (lectures and papers) and at least one you found yourself.
\item Discuss what is problematic and what is good about the definitions, illustrate this based on the cases discussed in the course.
\item Give three example requirements you took from official documents. Discuss to what extent these are true requirements and to what extent these follow the definitions.
\item Explain why a poorly described requirement statement can still be useful and why a well described requirement can be unuseful. 
\item Give an overview of requirements engineering.
\item List the different approaches, and their elements (steps/practices). At least one you found yourself.
\item Briefly discuss the key differences
\item What can you learn from studying the current domain about how things will work out in the future world that you are co creating?
\item Explain that RE is a form of truth finding.
\item Most methods in RE would qualify as rationalistic. Explain why a rationalistic approach is too unreliable based on one of the cases discussed in class. Explain how a scenario based approach adds not only certainty but also helps you better find requirements. 
\end{enumerate}

\section{Introduction}
In this portfolio, we are going to answer the questions presented in the questions section. In between, we will discuss different topics that are closely tied to the area of Requirements Engineering. Furthermore, we will tackle topics that are non technical, but still have a huge impact on the discipline.

\section{Software Requirements}

According to Jackson (1995), software development has many aspects, and many people see it differently. There is logic and mathematics, a socio-ethical challenge, field of labour employment relations, prolem in manufacturing process control, and finally, how I as a Software Engineering student see it is engineering - an activity of building useful things to serve recognizable purposes.

\subsection{Software}
Jackson (1995) states that to develop software is to build a MACHINE. The purpose of the machine is to make a change, meaning that introducing it into the world will have some effect there. This machine is part of a system.

The machine works in an environment. This environment will have an affect on the machine, and the machine will leave an affect on its environment. These parts of world are its APPLICATION DOMAIN (Jackson, 1995), and we will discuss them later on in the chapter dedicated to \hyperref[sec:domains]{Domains}

The problem lies in the application domain, and we use the machine as a solution (Jackson, 1995). 

With the previous said, we open a discussion topic which is one of the main points of the course Requirements Engineering, and a topic that we will be talking about throughout this portfolio. At the first glance, the machine can be a simple solution, but in complex systems, it never is. We should not let the intuition and heuristic lead us to effortless conclusions, and this will be also discussed in the \hyperref[sec:thinking]{Thinking} chapter.

The \hyperref[sec:thinking]{Thinking} chapter might also help us understand the focus of problems as Jackson (1995) states:

\textit{"  An inability to discuss problems explicitly has been one of the most glaring deficiencies of software practice and theory.
  Ralph Johnson, a leading advocate of design patterns in object-oriented development, has this to say:  'We have a tendency to focus on the solution, in large part because it is easier to notice a pattern in the systems that we build than it is to see the pattern in the problems we are solving that lead to the patterns in our solutions to them. '"}

Many products fail, even though during the development phase the projects were marked as a success, and this is noticed only after the work is done. As a result, this leads teams often to situations that we discuss, and that we don't want to be a part of. Multi million projects, with thousands of hours invested in it become a failure.

How is this possible?

Unfortunately, the answer is often found in requirements engineering. Because of that, this area of engineering carries a great significance in each and every project. Inadequately explored and described requirements increase the chance of a catastrophical outcome.



\section{Domains}
\label{sec:domains}

According to Jackson (1995) paying serious attention to the application domain means writing serious, explicit and precise DESCRIPTIONS. What everyone knows
is often wrong, and always very far from complete. 

Often, we see the parts of the solution (data flows, operations, use-cases, etc.) being part of the domain.

\subsection{Modeling}
In the Software Requirements and Specifications: A lexicon of Practice, Principles and Prejudices, Jackson states: 
\textit{"In principle you really need three descriptions: the common
description; the description· that's true only of the machine; and the
description that's true only of the application domain. If you make all of
those descriptions, and separate them carefully, you'll be all right."}


\section{Thinking}
\label{sec:thinking}



\section{Sources}
Jackson, M. (1995). Software Requirements and Specifications: A Lexicon of Practice, Principles and Prejudices (ACM Press) (Illustrated ed.). Addison-Wesley Professional.



\end{document}
